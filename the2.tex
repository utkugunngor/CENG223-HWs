\documentclass[12pt]{article}
\usepackage[utf8]{inputenc}
\usepackage{float}
\usepackage{amsmath}


\usepackage[hmargin=3cm,vmargin=6.0cm]{geometry}
%\topmargin=0cm
\topmargin=-2cm
\addtolength{\textheight}{6.5cm}
\addtolength{\textwidth}{2.0cm}
%\setlength{\leftmargin}{-5cm}
\setlength{\oddsidemargin}{0.0cm}
\setlength{\evensidemargin}{0.0cm}



\begin{document}

\section*{Student Information } 
%Write your full name and id number between the colon and newline
%Put one empty space character after colon and before newline
Full Name : Utku Gungor \\
Id Number : 2237477 \\

% Write your answers below the section tags
\section*{Answer 1}
\subsection*{1.a}
\hspace{15px} We can use \textit{($f^{-1}$} $\circ$ \textit{$g^{-1}$)}(\textit{$C_0$)} instead of \textit{$f^{-1}$}(\textit{$g^{-1}$(} \textit{$C_0$))} . \par 
If we show that \textit{(g $\circ$ f)} $\circ$ \textit{($f^{-1}$} $\circ$ \textit{$g^{-1}$)} = I and \textit{($f^{-1}$} $\circ$ \textit{$g^{-1}$)} $\circ$ \textit{(g $\circ$ f)} = I, it means the equation is true since the composition of a function and its inverse is equal to Identity. \par 
For the first one: \par 
\textit{(g $\circ$ f)} $\circ$ \textit{($f^{-1}$} $\circ$ \textit{$g^{-1}$)} = \textit{g} $\circ$ ((\textit{f} $\circ$ \textit{$f^{-1}$}) $\circ$ \textit{$g^{-1}$})    \hspace{20px}   (from Associativity Law) \par 
\hspace{108px} = \textit{g} $\circ$ (I $\circ$
\textit{$g^{-1}$)}  \par
\hspace{108px} = \textit{g} $\circ$ \textit{$g^{-1}$} \par 
\hspace{108px} = I \par 
For the second one: \par 
\textit{($f^{-1}$} $\circ$ \textit{$g^{-1}$)} $\circ$ \textit{(g $\circ$ f)} = ( \textit{$f^{-1}$} $\circ$ (\textit{$g^{-1}$} $\circ$ \textit{g))} $\circ$ \textit{f}  \hspace{20px}   (from Associativity Law) \par 
\hspace{108px} = \textit{($f^{-1}$} $\circ$ I) $\circ$ \textit{f} \par 
\hspace{108px} = \textit{$f^{-1}$} $\circ$ \textit{f} \par
\hspace{108px} = I \par 

Hence , \textit{(g $\circ$ f)$^{-1}$} (\textit{$C_0$}) = \textit{$f^{-1}$}(\textit{$g^{-1}$(} \textit{$C_0$))} for \textit{$C_0$} $\subseteq$ \textit{$C$} .\\
\subsection*{1.b}
\hspace{15px} Let's say A = \{$a_1$,$a_2$\} , B = \{$b_1$\} and C = \{$c_1$\} to determine whether f is injective or not.\par
First assume that f is not 1-to-1 . It means $\exists$ two members $a_1$ $\neq$ $a_2$ such that \textit{f($a_1$) = f($a_2$)= $b_1$} where $b_1$ $\in{B}$ . We know \textit{g $\circ$ f($a_1$)} and \textit{g $\circ$ f($a_2$)} means \textit{g(f($a_1$))} and \textit{g(f($a_1$))}.Then ,according to our assumption , $\exists$ \textit{g(f($a_1$))} = \textit{g(f($a_2$))} = $c_1$ where $c_1$ $\in{C}$ . Which means \textit{g $\circ$ f} is not 1-to-1 and there is a contradiction . Because we know \textit{g $\circ$ f} is injective(1-to-1) from the question. Therefore , \textit{f} must be injective . \par 
Now, let's assume A = \{$a_1$,$a_2$\} , B = \{$b_1$,$b_2$,$b_3$\} and C = \{$c_1$,$c_2$\} to determine whether g is injective or not.\par
There can be conditions such that : \par
\textit{f($a_1$) = $b_1$} \par 
\textit{f($a_2$) = $b_2$} \par
\textit{g($b_1$) = $c_1$} \par 
\textit{g($b_2$) = $c_2$} \par
\textit{g($b_3$) = $c_2$} \par 
Then , \textit{g $\circ$ f($a_1$) = $c_1$} and \textit{g $\circ$ f($a_2$) = $c_2$} . So , we'have seen that \textit{g $\circ$ f} is injective . Therefore , the function \textit{g} does not have to be injective .

\subsection*{1.c}
\hspace{15px} Let's say A = \{$a_1$,$a_2$\} , B = \{$b_1$,$b_2$,$b_3$\} and C = \{$c_1$,$c_2$\} and use the conditions of previous question :\par
\textit{f($a_1$) = $b_1$} \par 
\textit{f($a_2$) = $b_2$} \par
\textit{g($b_1$) = $c_1$} \par 
\textit{g($b_2$) = $c_2$} \par
\textit{g($b_3$) = $c_2$} \par 
It's clearly seen that f is not onto (surjective) in this example because of the member $b_3$. So , f does not have to be surjective. In addition , g is surjective in this assumption and if \textit{g $\circ$ f} is surjective , there is no condition that can make function \textit{g} non-surjective .Therefore , function \textit{g} must be surjective . 
\section*{Answer 2}

\subsection*{2.a}
\hspace{15px} If \textit{g} is left inverse for \textit{f} : \par 
Now , we need to show that if \textit{f(a)} = \textit{f(b)} , then a = b . \par 
If \textit{f(a)} = \textit{f(b)} , \textit{g(f(a))} = \textit{g(f(b))} .Then ,clearly \textit{g(f(a))} = \textit{a} and \textit{g(f(b))} = \textit{b} since \textit{g} is left inverse for \textit{f}. By comparing these two , we obtain \textit{g(f(a))} = \textit{a}  = \textit{b} = \textit{g(f(b))} . Therefore, f must be injective if it has a left inverse . \par \bigskip
If \textit{h} is right inverse for \textit{f} : \par
Now , we need to show that f is onto , i.e. for any member m $\in$ B , $\exists$ n $\in$ A with \textit{f(n)} = \textit{m} . \par 
If we choose \textit{h(m)} = \textit{n} , then we have \textit{f(n)} = \textit{f(h(m))} = \textit{m} since \textit{h} is right inverse for \textit{f}. Thus , f must be surjective if it has a right inverse .
\subsection*{2.b}
\hspace{15px} Let $\textit{f}$ : A $\rightarrow$ B and $\textit{g,h}$ : B $\rightarrow$ A with A = \{1,2\} , B = \{a,b,c\} . \par 
Take \textit{f(1) = a} , \textit{f(2) = b} ; \textit{g(a) = 1} , \textit{g(b) = 2} , \textit{g(c) = 1} ; \textit{h(a) = 1} , \textit{h(b) = 2} ,\textit{h(c) = 2} . It's clearly seen that \textit{h} $\circ$ \textit{f} = \textit{g} $\circ$ \textit{f} = $\iota_A$ but $\textit{g}$ $\neq$ $\textit{h}$ . We've showed that there are two left inverses of $\textit{f}$ in this example . Hence , a function can have more than one left inverse . \par \bigskip
Let's explain the question for right inverses by giving another example . \par 
Say \textit{f} : \{a,b\} $\rightarrow$ \{c\} . In this case , \textit{f(a)} = \textit{f(b)} = \textit{c} . Now we can define two different right inverses of \textit{f} , say \textit{g} and \textit{h} . It is obvious that \textit{g} : \{c\} $\rightarrow$ \{a,b\} and \textit{h} : \{c\} $\rightarrow$ \{a,b\} . There can be condition such that \textit{g(c)} = \textit{a} and \textit{h(c)} = \textit{b} . These functions are both right inverse for \textit{f} and they are different from each other . Thus , a function can have more than one right inverse . 
\subsection*{2.c}
\hspace{13px} Assume \textit{f} has left and right inverse. \par   If \textit{f} has left inverse \textit{g} , then \textit{f} is injective which is proven in part(a) . Similarly , if \textit{f} has right inverse \textit{h} , then \textit{f} is surjective from part(a) again . Thus , as \textit{f} is both injective and surjective , it is bijective . If \textit{f} is a bijective function , its right and left inverses must be same . So , g = h = \textit{$f^{-1}$} . 
\section*{Answer 3}
\subsection*{For the function f}
\hspace{15px} We know \textit{f} is defined on \textit{$Z^{+}$} . For the second parameter which is \textit{y} , we can easily obtain any number we want on this region since it depends only on \textit{y} , so the second parameter is surjective . Now , let's consider the conditions \textit{y = 1} and \textit{y = 2} . For \textit{y = 1} , our function becomes ; \par  \textit{f(x , 1)} = \textit{(x + 1 - 1 , 1)} = (\textit{x} , 1) \par 
And x $\in$ \textit{$Z^{+}$} . If \textit{x = 1} , we have (1 , 1) ; if \textit{x = 2} , we have (2 , 1) ; if \textit{x = 3} , we have (3 , 1) and so on . \par 
For \textit{y = 2} , our function becomes ; \par 
\textit{f(x , 2)} = \textit{(x + 2 - 1 , 2)} = (\textit{x + 1} , 2) \par If \textit{x = 1} , we have (2 , 2) ; if \textit{x = 2} , we have (3 , 2) ; if \textit{x = 3} , we have (4 , 2) and so on . \par The condition y $\leq$ x is satisfied and we can obtain numbers in \textit{$Z^{+}$} , hence first parameter is surjective , too . Since both parameters are surjective , \textit{f} is surjective . \par 
For the injectivity of \textit{f} , the second parameter depends only on \textit{y} , so there is only one value for each \textit{y} and this parameter is clearly injective . Then , let's think some \textit{y} is chosen which means first parameter depends only on \textit{x} . So , there is only one value for each \textit{x} , hence first parameter is injective ,too . Therefore , \textit{f} is injective. \par 
Since \textit{f} is both surjective and injective , it is bijection .

\subsection*{For the function g}
\hspace{15px} Let's try some values of \textit{x} and \textit{y} to determine whether it is surjective . \par 
\textit{g(1 , 1)} = 1 \par 
\textit{g(2 , 1)} = 2 \par 
\textit{g(2 , 2)} = 3 \par 
\textit{g(3 , 1)} = 4 \par 
These values satisfy the condition y $\leq$ x and x,y $\in$ \textit{$Z^{+}$} . We can easily obtain all numbers $\in$ \textit{$Z^{+}$} . Hence , \textit{g} is surjective . For the injectivity , we can compare some values of the function \textit{g} .\par 
\textit{g(1 , 1)} = 1 \par 
\textit{g(2 , 1)} = 2 \par 
\textit{g(2 , 2)} = 3 \par 
\textit{g(3 , 1)} = 4 \par 
\textit{g(3 , 2)} = 5 \par 
\textit{g(3 , 3)} = 6 \par 
These values satisfy the condition y $\leq$ x and x,y $\in$ \textit{$Z^{+}$} again . We could think that values we've tried can give same images since they are close numbers . But all of them gave different images . While \textit{x} and \textit{y} are increasing , the image of \textit{g} is increasing ,too . Hence , \textit{g} is injective . \par 
As \textit{g} is both surjective and injective , it is bijection .

\section*{Answer 4}

\subsection*{4.b}
\hspace{15px} I couldn't prove the part (a) . But it says the set of algebraic numbers is countable. So , we can use this . \par 
Let's say A for the set of algebraic numbers and B for the set of transcendental numbers . And R = A $\cup$ B . We know that R is uncountable . So one of A or B must be uncountable . A is countable from part(a) . Therefore, B is uncountable .

\section*{Answer 5}

\hspace{13px} We can choose n for k . Now , we should show that \textit{n}ln\textit{n} is $O(n)$ and \textit{n}ln\textit{n} is $\Omega(n)$ . \par 
When n $\leq{e} $ , $\mid \textit{n}$ ln\textit{n} $\mid$ $\leq$ C $\mid n \mid $ for C = 1 .\par 
\hspace{30px} n $\geq$ e , $\mid \textit{n}$ ln\textit{n} $\mid$ $\geq$ C $\mid n \mid $ for C = 1 . \par
Therefore , \textit{n} ln\textit{n} = $\Theta(k)$ for k = n .  \par \bigskip
Now , it's time to show that n = $\Theta(\frac{n}{ln\textit{n}})$ because we are using \textit{n} instead of \textit{k} . \par 
For n $\leq{e} $ , $\mid \textit{n}\mid$ $\leq$ C $\mid \frac{n}{ln\textit{n}} \mid $ for C = 1 .\par 
\hspace{17px} n $\geq{e} $ , $\mid \textit{n}\mid$ $\geq$ C $\mid \frac{n}{ln\textit{n}} \mid $ for C = 1 .\par 
We have showed that n = $\Theta(\frac{n}{ln\textit{n}})$ . Hence , n = $\Theta(\frac{k}{ln\textit{k}})$ .


 
\section*{Answer 6}

\subsection*{6.a}

\hspace{15px} Positive divisors of 6 which are other than itself are 1,2,3 . And 6 = 1+2+3 . \par \bigskip
Positive divisors of 28 that are other than itself are 1,2,4,7,14 . And 28 = 1+2+4+7+28 . \par \bigskip
Therefore , as they satisfied the definition of perfect number , 6 and 28 are perfect number . \bigskip

\subsection*{6.b}

\hspace{15px} Positive divisors of \textit{$2^{p-1}$}(\textit{$2^{p} - 1$} ) other than itself are all the number with the type \textit{$2^{a}$} for 0 $\leq$ a $\leq$ p - 1 , and \textit{$2^{b}$}(\textit{$2^{p}$} - 1) for 0 $\leq$ b $\leq$ p - 1 when \textit{$2^{p}$} - 1 is prime . And we can obtain  $\sum_{a=0}^{n}$\textit{$2^{a}$} = \textit{$2^{n+1} - 1$} by using the formula for $\sum_{j=0}^{n}$ \textit{a}\textit{$r^{j}$} from the text book (n $\in$ \textit{$Z^{+}$}) . \par 
Let's take the sum of these positive divisors : \par \bigskip
$\sum_{a=0}^{p - 1}$\textit{$2^{a}$} + $\sum_{b=0}^{p - 2}$ \textit{$2^{b}$}(\textit{$2^{p} - 1$}) = \textit{($2^{p} - 1$)} + \textit{($2^{p} - 1$)} \textit{($2^{p - 1} - 1$)}  \par 
\hspace{136px}  = \textit{($2^{p} - 1$)}\textit{(1 +} \textit{($2^{p - 1} - 1$))} \par 
\hspace{136px} = \textit{($2^{p} - 1$)}\textit{$2^{p - 1}$} \par 
We've reached our first expression . Hence , when \textit{($2^{p} - 1$)} is prime , \textit{$2^{p-1}$}(\textit{$2^{p} - 1$} ) is a perfect number .




\section*{Answer 7}

\subsection*{7.a}

\hspace{15px} From given expressions ,  we can say \textit{x} = \textit{a.m} + \textit{$c_1$} = \textit{b.n} + \textit{$c_2$} where \textit{a,b,m,n,} \textit{$c_1$},\textit{$c_2$} $\in$ Z with m $>$ 0 and n $>$ 0 . Then , we have : \par \bigskip

$\rightarrow$ \textit{$x$} \textit{$-$} \textit{$c_1$ =} \textit{a.m} \par
$\rightarrow$ \textit{$x$} \textit{$-$} \textit{$c_2$ =} \textit{b.n} \par
Now , let's assume \textit{gcd(m,n) = k} . 
And clearly ;  \par 
$\rightarrow$ \textit{k} $\vert$ \textit{(a.m)} and \textit{k} $\vert$ \textit{(b.n)} \par 
$\rightarrow$ \textit{k} $\vert$ \textit{(x} \textit{$- c_1)$} and \textit{k} $\vert$ \textit{(x} \textit{$- c_2)$}\par 
$\rightarrow$ \textit{(x} \textit{$- c_2) - $} \textit{(x} \textit{$- c_1)$} = \textit{(k.u)} \textit{$-$} \textit{(k.v)} for some u,v $\in$ Z \par 
$\rightarrow$ \textit{$(c_1 - c_2)$} = \textit{$k (u - v)$} \par 
$\rightarrow$ \textit{k} $\vert$ \textit{$(c_1 - c_2)$} \par 
We proved the condition \textit{gcd(m,n)} $\vert$ \textit{$(c_1 - c_2)$} .

\end{document}

​

